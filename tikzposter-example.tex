%%
%% This is file `tikzposter-example.tex',
%% generated with the docstrip utility.
%%
%% The original source files were:
%%
%% tikzposter.dtx  (with options: `tikzposter-example.tex')
%% 
%% This is a generated file.
%% 
%% Copyright (C) 2014 by Pascal Richter, Elena Botoeva, Richard Barnard, and Dirk Surmann
%% 
%% This file may be distributed and/or modified under the
%% conditions of the LaTeX Project Public License, either
%% version 2.0 of this license or (at your option) any later
%% version. The latest version of this license is in:
%% 
%% http://www.latex-project.org/lppl.txt
%% 
%% and version 2.0 or later is part of all distributions of
%% LaTeX version 2013/12/01 or later.
%% 








 \documentclass[25pt, a0paper, landscape, margin=0mm, innermargin=15mm,
     blockverticalspace=15mm, colspace=15mm, subcolspace=8mm]{tikzposter} %Default values for poster format options.
% \usepackage{lmodern}
% \makeatletter
% \input{theguy36pt.clo}
% \makeatother
% \usepackage{blindtext}
% \block{Walzing Wombat}{\blindtext}

 \tikzposterlatexaffectionproofon %shows small comment on how the poster was made at bottom of poster

 % Commands
 \newcommand{\bs}{\textbackslash}   % backslash
 \newcommand{\cmd}[1]{{\bf \color{red}#1}}   % highlights command

 % Title, Author, Institute
 \title{LibRoadRunner: High-Performance Timecourse Simulation and Model Fitting}
 \author{J. Kyle Medley, Wilbert Copeland, Kiri Choi, Kaylene Stocking and Herbert M. Sauro}
 \institute{University of Washington, Seattle WA}
 \titlegraphic{\includegraphics[scale=4.0]{SignatureLeftPurple.pdf}}

 % -- PREDEFINED THEMES ---------------------- %
 % Choose LAYOUT:  Default, Basic, Rays, Simple, Envelope, Wave, Board, Autumn, Desert,
 \usetheme{Autumn}
\definecolor{mybannercolor}{rgb}{0.13, 0.055, 0.063}
\definecolor{mybgcolor}{RGB}{232, 229, 206}
\usecolorstyle[colorPalette=BrownBlueOrange,colorTwo=mybgcolor]
{Germany}

\usepackage{pgfplots}
\usepackage{verbatim}
\usepackage{listings}
\usepackage{tabularx}

\usetikzlibrary{arrows,positioning}

 \begin{document}

     \maketitle

     \begin{columns}%blocks will be placed into columns
         \column{.55}
         \block[roundedcorners=40]{Introduction}{
          {
%             \Large
            Timecourse simulation of kinetic models is an extremely important tool
            in biology, and has found applications in pharmacology (PK/PD models),
            rational design of synthetic systems (Elowitz, \textit{Nature} 2000),
            and whole-cell model simulation (Karr, \textit{Cell} 2012).
            Pushing the envelope of high-speed model simulation is key to exploring
            more complex and diverse biological modeling endeavors.

            libRoadRunner is designed to be a high-performance simulator and analysis
            library for SBML-encoded models.
            To facilitate maximum performance, libRoadRunner makes use of the LLVM library
            (\texttt{llvm.org}) to compile models down to native code which can be
            executed directly on the target CPU.

            By using an open and transparent development process, making our source code
            freely available, and working extensively with our collaborators through
            one-week workshops, we have enabled widespread access to this powerful
            modeling tool.
          }

          \vspace{50pt}

          \begin{tikzfigure}[The libRoadRunner native compilation pipeline]
            \begin{tikzpicture}[->,>=stealth',very thick]
              \node (sbmllogo) {\includegraphics[scale=6.0]{sbml-logo.pdf}};
              \node (ircode)[right= 3 of sbmllogo, align=left] {
              \small\texttt{define internal double}\\
              \small\texttt{@getFloatingSpeciesConcentration}\\
              \small\texttt{(\%rr\_LLVMModelData* \%modelData, i32 }\\
              \small\texttt{entry: switch i32 \%floatingSpeciesIndex}
              };
              \node (machinecode)[right= 3 of ircode, align=left] {
              \small\texttt{  7c2b20:  55                    push   \%rbp}\\
              \small\texttt{  7c2b21: 48 89 e5              mov    \%rsp,\%rbp}\\
              \small\texttt{  7c2b2e: 5d                    pop    \%rbp}\\
              \small\texttt{  7c2b2f: c3                    retq}
              };
              \node (simplot)[right= 3 of machinecode] {
                \begin{tikzpicture}
                  \begin{axis}[
                    width=0.10\textwidth,
                    height=0.09\textheight,
                    xlabel=$t$,
                    domain=0:1,
                    xtick=\empty,%{0,0.25,0.75,1},
                    ytick=\empty,
                    axis lines = middle,
                    enlargelimits = true,
                    axis line style = very thick] %http://tex.stackexchange.com/questions/237709/pgfplots-ignores-thick-option-for-axes
                    \addplot[smooth,mark=*,blue] {x^2*(1+sin(deg((x*5)^2)))};
                    \addlegendentry{$f(t,p)$}
                  \end{axis}
                \end{tikzpicture}
              };

              \path
                (sbmllogo) edge[line width=5pt] node [left] {} (ircode)
                (ircode) edge [right,line width=5pt] node[left] {} (machinecode)
                (machinecode) edge [right,line width=5pt] node[left] {} (simplot) ;

              \node [below= of sbmllogo] {\textbf{SBML Model Format}} ;
              \node [below= of ircode] {\textbf{LLVM IR Code}} ;
              \node [below= of machinecode] {\textbf{Native Machine Code}} ;
              \node [below= of simplot] {\textbf{Simulation}} ;
            \end{tikzpicture}
          \end{tikzfigure}

          \vspace{50pt}

          \innerblock[]{Native Performance}{libRoadRunner uses a non-tracing JIT to run simulations with the speed of a custom, model-specific executable}
%             \coloredbox{Text may be highlighted using colored boxes created by \bs\texttt{coloredbox[{\it options}]\{{\it Text\}}}}

     }
     \block{Blocks}{
         Blocks are arranged in a grid, by default, with width by default \texttt{\bs textwdith}.  They are created by the command
         \begin{quote}
             \bs\texttt{block [{\it options}] \{{\it title}\}\{{\it contents}\}}
         \end{quote}
         The title may be left empty, resulting in no title area being created for the block (as seen in a later block to the right).  Further blocks will be placed below automatically, at a distance defined by \texttt{blockverticalspace}.

         If you want to change the position of the title matter or the contents in the block, you may by setting in the options
         \begin{quote}
             \texttt{titleoffsetx, titleoffsety, bodyoffsetx, bodyoffsety}
         \end{quote}
         which let you adjust the vertical or horizontal position of the two parts of the block, respectively.  You can also make, relative to the default width, the title and block body by setting
         \begin{quote}
             \texttt{titlewidthscale, bodywidthscale}
         \end{quote}
         The title's alignment can be set by \texttt{titleleft, titlecenter, titleright}, the body may be shifted vertically by setting \texttt{bodyverticalshift}, and the shape of the block can be altered by setting \texttt{roundedcorners, linewidth}. The inner margins of the title can by set by \texttt{titleinnersep,bodyinnersep}.
     }

%      \note[targetoffsetx=24cm, targetoffsety=-9cm,rotate=1,angle=270,radius=8cm,width=.75\textwidth,innersep=.4cm]{
%          You can place notes that are ``attached'' to the previous block using the command
%          \begin{quote} \texttt{\bs note[{\it options}]\{{\it contents}\}}\end{quote}
%          The note is placed by default slightly to the right of a ``target'' in the center of the previous block.  The note style may also allow for a connection between the note and the ``target''.  \\
%          The target may be shifted from the default by setting the options  \texttt{targetoffsetx, targetoffsety}, rotated by an angle with \texttt{rotate}, and its width with \texttt{width}.  The placement of the note in relation with the target is given in polar coordinates with \texttt{ radius, angle}. Please observe that notes are always drawn {\bf over} the other objects. They do not affect the placement of blocks.
%       }

    \column{.45}
%      \note[targetoffsetx=-.05\textwidth,targetoffsety=9.5cm,innersep=.4cm,angle=-45,connection]{Optional arguments for the format of the poster}
      \block{Under the Hood}{
          Our C++ library features a modular design, allowing effortless upgrades to its internal architecture.
          The library features pure, abstract base classes for \textbf{integrators} and \textbf{steady-state solvers}.
          Using these interfaces, we have implemented a timecourse integrator (\textit{CVODE}) from the
          Lawrence Livermore National Laboratory's \textit{SUNDIALS} suite.
          CVODE features an Adams-Moulton method for non-stiff problems, and a Backwards Differentiation
          Formula (BDF) in fixed-leading coefficient form for stiff problems. %https://computation.llnl.gov/casc/sundials/description/description.html
          For locating steady states, libRoadRunner uses the NLEQ library (Nowak and Weimann, 1991),
          a damped Newton method for highly nonlinear equations.
          \innerblock[]{APIs}{libRoadRunner features APIs for C, Python, and Java, enabling it to be used from any of these languages}
      }
%          \block{Columns}{
%               By default, blocks are arranged in a single column. If you want multiple columns for your poster, you may use the \texttt{columns} environment. For example,
%              \begin{quote}
%                  \texttt{\noindent \bs begin\{columns\}\\
%                  \bs column\{.6\}\\
%                  \bs block\{\dots\}\{\dots\}\\
%                  \bs column\{.4\}\\
%                  \bs block\{\dots\}\{\dots\}\\
%                  \bs block\{\dots\}\{\dots\}\\
%                  \bs end\{columns\}
%                  }
%              \end{quote}
%              will create two columns of 60\% and 40\% the available width; spacing between successive columns is handled automatically.  The block command(s) following \texttt{\bs column} are the blocks to go in that column.  The number of columns is free to be chosen, but the relative widths must all be chosen.  If the widths sum to less than 1, empty space will be seen on the right.  If they sum to more than 1, the latter columns will be cut off.
%          }

         \begin{subcolumns}
             \subcolumn{.45}
             \block{Subcolumns}{If you want to have an additional subdivision of columns inside a column, you may use the\\ \texttt{\bs subcolumns} environment inside of a column environment.  The functionality is similar to that of columns, but now the widths are relative to the width of the current column.}

             \subcolumn{.5}
             \block{}{An example use of subcolumns is.
                 \begin{quote}
                     \texttt{\bs begin\{subcolumns\}\\
                     \bs subcolumn\{.6\}\\
                     \bs block\{\dots\}\{\dots\}\\
                     \bs subcolumn\{.4\}\\
                     \bs block\{\dots\}\{\dots\}\\
                     \bs block\{\dots\}\{\dots\}\\
                     \bs end\{subcolumns\}
                     }
             \end{quote}
         }
         \end{subcolumns}

%          \block[titlewidthscale=.8,bodywidthscale=.9,titleoffsety=9.5mm,bodyoffsety=9mm]{Changing the Poster's Appearance}{
         \block{Acknowledgments and Funding Sources}{
             We wish to acknowledge Andy Somogyi for his extensive work on libRoadRunner
             (in particular, for writing the LLVM simulation engine).
             We also acknowledge Totte Karlsson for the original C\# to C++ translation, C compiler backend and C API, Stanley Gu for testing the library as a web service, Lucian Smith for developing part of the test suite, Michael Galdzicki for writing detailed build instructions and testing for developers.
             This work was funded by generous support from NIH grant R01 GM081070. The content is solely the responsibility of the authors and does not necessarily represent the views of the National Institutes of Health.

             \vspace{2em}

             Logos are copyright their respective owners.
          }

     \end{columns}

     \block[]{}{
%      \block[titleoffsety=-1cm,bodyoffsety=-1cm]{Sample document}{
      \resizebox{0.95\textwidth}{!}{ %http://tex.stackexchange.com/questions/121155/how-to-adjust-a-table-to-fit-on-page
        \includegraphics[scale=0.35]{libroadrunner_logo_tan.jpg}
        \hspace{10em}
        \includegraphics[scale=5.0]{sbml-logo.pdf}
        \hspace{10em}
        \includegraphics[scale=0.8]{SEDMLlogo.png}
        \hspace{10em}
        \includegraphics[scale=0.17]{LLVMLogoTight.pdf}
        \hspace{10em}
        \includegraphics[scale=5.0]{sbml-logo.pdf}
        \hspace{10em}
        \includegraphics[scale=5.0]{sbml-logo.pdf}
      }
%      \vspace{2em}
%          This poster was created by the following commands (omitting the contents of the blocks and notes) to give a sense of how different objects are created and options used.
%          \begin{quote}
%              \texttt{\bs documentclass[25pt, a0paper, landscape, margin=0mm, innermargin=15mm,
%          blockverticalspace=15mm, colspace=15mm, subcolspace=8mm]\{tikzposter\}\\
%              \bs title\{Using tikzposter\} \bs author\{Pascal Richter, Elena Botoeva, Richard Barnard, \& Dirk Surmann\} \bs institute\{\}\\
%               \bs usetheme\{Autumn\}\bs usecolorstyle[colorPalette=BrownBlueOrange]\{Germany\}\\
%              \bs begin\{document\}\bs maketitle\\
%              \bs begin\{columns\} \bs column\{0.55\}\\
%              \bs block\{Creating the document\}\{The document\dots\} \bs note[targetoffsetx=-.05\bs textwidth,targetoffsety=9.5cm,innersep=.4cm,angle=-45,connection]\{\dots\}\\
%              \bs block\{The title matter\}\{The title\dots\}\\
%              \bs block\{Blocks\}\{Blocks are\dots\} \bs note[targetoffsetx=24cm, targetoffsety=-9cm,rotate=1,angle=270,radius=8cm,width=.75\bs textwidth,innersep=.4cm]\{You can\dots\}\\
%              \bs column\{0.45\} \bs block\{Columns\}\{By default,\dots\}\\
%              \bs begin\{subcolumns\} \bs subcolumn\{.45\}
%              \bs block\{Subcolumns\}\{If you\dots\}
%              \bs subcolumn\{.5\} \bs block\{\}\{An example\dots\}
%              \bs end\{subcolumns\}\\
%              \bs block[titlewidthscale=.8,bodywidthscale=.9,titleoffsety=9.5mm,bodyoffsety=9mm]\{Changing the Poster's Appearance\}\{If the default\dots\}
%              \bs end\{columns\}\\
%              \bs block[titleoffsety=-1cm,bodyoffsety=-1cm]\{Sample document\}\{This poster\dots\}\\
%              \bs end\{document\}
%              }
%          \end{quote}
     }

 \end{document}




\endinput
%%
%% End of file `tikzposter-example.tex'.
